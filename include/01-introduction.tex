Maxwell equations, the equations describing the behaviour of the electromagnetic field, come with rather startling implications. One of the implications becomes aparent, when considering the Maxwell equations for vacuum (setting charge and current to $0$) and substituting from one equation into another, one can produce the well-known {\it wave equation}, meaning that disturbances in the electromagnetic field are able to propagate though otherwise empty space as a wave. The wave equation also predicts the speed at which this wave should propagate and which turns out to be the speed of light, directly hinting to the fact, that light itself is an electromagnetic wave.

Now as is the case with Maxwell equations, the Einstein equations (\ref{ee}) can be manipulated to produce a wave equation for a small perturbation $h_{\mu\nu}$ in a nearly flat spacetime given by metric tensor $g_{\mu\nu} = \eta_{\mu\nu}+h_{\mu\nu}$, showing that disturbances in a metric tensor can propagate as a waves too. 

\begin{equation}\label{ee}
    R_{\mu\nu} - \frac{1}{2}g_{\mu\nu}R = \frac{8\pi G}{c^4} T_{\mu\nu}
\end{equation}

The Einstein equations quantify a relationship between a metric tensor $g_{\mu\nu}$ (Ricci tensor $R_{\mu\nu}$ and Ricci scalar $R$ are both functions of $g_{\mu\nu}$ and it's derivatives), which quantifies how distances in space and time should be measured, and the energy-momentum tensor $T_{\mu\nu}$ containing the energy and momentum contents of space. In other words, it's according to this equation (\ref{ee}) that matter tells spacetime how to curve and spacetime tells matter how to move.

